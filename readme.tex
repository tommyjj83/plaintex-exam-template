\documentclass{article}

\usepackage[left=1in,right=1in,bottom=1in, top=1in]{geometry}
\usepackage{fontspec}
\usepackage{polyglossia}
\setmainlanguage{czech}

\title{Šablona pro zadání testů v Plain\TeX{}u}
\author{Tomáš Jaroš}
\date{\today}

\begin{document}
\maketitle
\tableofcontents
\newpage

\section{Zadání}
    Tato práce má za cíl vytvořit jednoduchou šablonu pro zadávání testů v Plain\TeX{}u. Pokryje základní typy otázek, které se mohou v testech
    vyskytnout, a poskytne uživateli základní nástroje pro tvorbu zadání. Návrh je realizován pomocí maker a uživateli je poskytnuta
    sada maker pro sázení zadání.

\section{Poskytnutá makra}
\subsection{Hlavička}
    Uživatel si může zvolit libovolný počet a kombinaci ze čtyř maker, které slouží pro sázení hlavičky zadání.

    První nabízené makro je \texttt{\textbackslash setcourse\{\}}, které bere jako svůj parametr
    název kurzu. Ten je zobrazen velkým tučným písmem a je vyrovnán na střed.

    Další poskytnuté makro je \texttt{\textbackslash setvariant\{\}}. Jediným parametrem
    je varianta testu, např. A či 1. Varianta je zobrazena v levém hodním rohu na úrovni kurzu.

    Třetí nabízené makro je \texttt{\textbackslash includenameinheader}. Toto makro způsobí,
    že se v záhlaví objeví volná kolonka pro jméno zkoušeného. To je téměř nedílnou součástí
    každého zadání.

    Poslední z těchto maker je \texttt{\textbackslash includescoring}. Toto makro vytvoří tabulku,
    do které je možné zapsat získané body z jednotlivých otázek. To pomůže zkoušejícímu zpřehlednit
    hodnocení a vyvarovat se chybám. Toto makro se adaptuje na počet zadaných otázek, z toho důvodu
    funguje až při druhém průchodu. Tabulka je vytvořena dle typografických pravidel, tedy
    nechána téměř volná, bez mříží.

    Hlavička se vypíše až po použití makra \texttt{\textbackslash printheader}. Jak bylo zmíněno, uživatel
    si může zvolit libovolnou kombinaci ze zmíněných čtyř maker. Implementace se postará,
    aby byly vloženy vhodné mezery, či nebyla vložena žádná, pokud to aktuální výběr nevyžaduje.
    Vždy je tak udržena přehlednost a zároveň nerušivost těchto prvků.
    Po hlavičce následuje hlavní část, je zde proto vložena typografická linka pro odělení těchto
    částí testového papíru.

\subsection{Otázky}
    Návrh podporuje několik typů otázek, které pokrývají všechny potřebné scénáře použití. Důraz byl kladen
    na jejich úplnost, jednoduchost použití a typografickou korektnost.

\subsubsection*{Otázky s možnostmi}
    Prvním skupinou otázek jsou otázky s možnostmi. Jako první je otázka s jedinou správnou odpovědí,
    tedy tzv. single-choice. Uživatel započne tento typ otázky makrem \texttt{\textbackslash singlechoicequestion\{\}\{\}}.
    Jako první se předává samotná otázka, druhým parametrem je pak počet bodů za tuto otázku.
    Jednotlivé možnosti se specifikují makrem \texttt{\textbackslash choice\{\}}, jehož jediným
    parametrem je právě text možnosti.
    Obdobně fungují i otázky s více možnými odpověďmi, tzv. multi-choice, které započnou makrem \texttt{\textbackslash multichoicequestion\{\}\{\}},
    a otázky typu pravda/nepravta, tzv. true-false, které započnou makrem \texttt{\textbackslash truefalsequestion\{\}\{\}}.
    Nakonec pro ukončení uživatel zadá \texttt{\textbackslash endchoicequestion}, které mimojiné vizuálně oddělí tuto
    otázku od následující. Poslední zmíněné makro je společné pro všechny otázky s možnostmi.

    U tohoto typu otázek byl kladen důraz na tvar a velikost boxů pro zaškrtávání. Tvar odlišuje, o který typ
    otázky se jedná, nedojde tak ke zmatení a typ otázky nemusí být explicitně zmíněn. Velikost je uspůsobená
    pro přehledné označení. Pozice boxů je navíc volena tak, aby zkoušeného nerozptylovala. Je zarovnána pod zadání, nikterak nevyčnívá
    a při pohledu na celý list jednotlivé boxy vypají jako součást textu.

    \subsubsection*{Otázky s volnými odpověďmi}
    Druhým typem nabízených otázek jsou otázky s volnými odpověďmi. Jednoduchá otázka tohoto typu je
    \texttt{\textbackslash answerquestion\{\}\{\}\{\}}. Toto makro požaduje tři parametry: samotnou otázku,
    počet bodů za otázku a požadovaný prostor pro odpověď. Otázka a prostor pod ní je brána jako jeden celek,
    proto pokud by se společně nevešli na stránku, jsou přesunuty na následnující. Návrh s tím počítá a
    v takovém případě jsou automaticky nataženy mezery mezi jednotlivými otázkami. Nedochází tak k underfull chybě.

    Makro zmíněné v předchozím odstavci lze použít i pro otázku s podotázkami. Stačí před seznam takovýchto podotázek
    vložit makro \texttt{\textbackslash subquestions\{\}\{\}}, očekávající zadání a počet bodů, a tento
    seznam ukončit makrem \texttt{\textbackslash endsubquestions}. Jednotlivé otázky pak nejsou značeny čísly, ale písmeny, a
    jsou zobrazeny jako podotázky.

    Délka zadání otázek i možností není limitována na délku řádky. Návrh dokáže delší text zalomit podle
    aktuálního prostoru (který se pro každý typ otázky liší, viz velikost boxů true-false otázek a single-choice otázek). 

\section{Závěr}
    V této práci jsem vytvořil plně použitelnou šablonu pro vytváření testových zadání. Návrh šablony podporuje
    různé typy otázek, které pokrývají potřeby zkoušejícího. Zadání otázek jsou konfigurovatelná a poradí si s textem libovolné
    délky. Dostupná jsou také makra pro vytvoření hlavičky, která je často nedílnou součástí testů. Ta se umí automaticky
    přizpůsobit počtu otázek či zvoleným možnostem a upravit tak svůj vzhled. Návrh respektuje typografická pravidla
    a neruší zkoušeného od přemýšlení.

    Při implementaci byla vynaložena snaha pro jednoduchost použití. Z toho důvodu se mohou některá makra použít stejným způsobem
    v různém kontextu a docílit tak jiných výsledků. Použítí je však stále intuitivní.

    Makra jsou využitelná pro Plain\TeX{} a mohou být zpracována libovolným moderním enginem, nevyužívají
    totiž žádných speciálních konstrukcí (např. \texttt{\textbackslash directlua}). Samotná ukázka je zpracována v Lua\TeX{}u.  

\end{document}
