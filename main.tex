% Semestral work for class BI-TEX - template for exams

\setbox0=\hbox{\bf Q1: }
\parindent=\wd0 % So that when "Qn: " is written, it would start nicely below

\font\courseTitle=cmb10 at 17pt

\hoffset=-18pt
\voffset=-18pt
\hsize=\dimexpr597pt-54pt-54pt
\vsize=\dimexpr845pt-54pt-54pt

\footline{}


%%%%%%%%%%%%%%%%%%%%%%%%%%%%%%%%%%%%%%%%%%%%%%%%%%
% Variables begin
%%%%%%%%%%%%%%%%%%%%%%%%%%%%%%%%%%%%%%%%%%%%%%%%%%

\newcount\questionCnt\questionCnt=1  % Question numbers start at 1
\newcount\subquestionCnt\subquestionCnt=`a  % Subquestions start at a
\newcount\i % Help in loops

\newwrite\countFileOut
\newread\countFileIn

\def\questionSingle{single}
\def\questionMultiple{multi}
\def\questionTF{tf}
\def\questionSub{sub}
\def\questionType{none}

%%%%%%%%%%%%%%%%%%%%%%%%%%%%%%%%%%%%%%%%%%%%%%%%%%
% Variables end
%%%%%%%%%%%%%%%%%%%%%%%%%%%%%%%%%%%%%%%%%%%%%%%%%%


%%%%%%%%%%%%%%%%%%%%%%%%%%%%%%%%%%%%%%%%%%%%%%%%%%
% Definition of internal macros begin
%%%%%%%%%%%%%%%%%%%%%%%%%%%%%%%%%%%%%%%%%%%%%%%%%%

\def\centerOptionMark#1{\hbox to 40pt{\hss#1\hss}}

\def\singleChoiceBox{%
    \raise 1pt\hbox{%
        $\bigcirc$
    }%
}

\def\multiChoiceBox{%
    \lower 1.7pt
    \hbox{%
        \vrule width 0.4pt height 10pt
        \vbox{%
            \hrule width 9.2pt height 0.4pt
            \vskip 9.2pt
            \hrule width 9.2pt height 0.4pt
        }%               
        \vrule width 0.4pt height 10pt
    }%
}

\def\TFBox{%
    \hbox{%
        \multiChoiceBox
        \hskip 3pt
        \multiChoiceBox
    }%
}

\def\question#1#2{%
    \setbox0=\hbox{
        \ifx\questionType\questionSub
            \hskip\parindent\char\subquestionCnt)
        \else
            \noindent
            {\bf Q\the\questionCnt: }%
        \fi
    }%
    \hbox{%
        \copy0
        \vtop{%
            \noindent\hsize=\dimexpr\hsize-\wd0
            #1\ifx#2\empty\else~(#2\thinspace pt)\fi
        }%
    }%
    \vskip 5pt
}

\def\answerQuestion#1#2#3{%
    \vbox{%
        \question{#1}{#2}%
        \vskip #3 plus 1fil
    }%
}

\def\saveQuestionCnt{%
    \immediate\openout\countFileOut=questioncnt.dat
    \immediate\write\countFileOut{\the\questionCnt}%
    \immediate\closeout\countFileOut
}

\def\printScoring{%
    \openin\countFileIn=questioncnt.dat
    \ifeof\countFileIn
        \questionCnt=0
    \else
        \read\countFileIn to \x
        \closein\countFileIn
        \questionCnt=\x
    \fi
    \hbox to \hsize{
        \hss
        \vbox to 3em{%
            \vss
            \hbox{Question\hfil}%
            \vskip 10pt
            \hbox{Points\hfil}%
            \vss
        }%
        \hskip 5pt
        \vrule\hskip 1pt\vrule
        \hskip 5pt
        \i=0
        \loop
            \advance\i by 1
            \vbox to 3em{%
                \vss
                \hbox to 1cm{\hfil\the\i\hfil}%
                \vskip 10pt
                \hbox to 1cm{\hfil\phantom{P}}%
                \vss
            }%
            \ifnum\i<\questionCnt
                \repeat
        \hss
    }
    \questionCnt=1  % Set back to 1
}

%%%%%%%%%%%%%%%%%%%%%%%%%%%%%%%%%%%%%%%%%%%%%%%%%%
% Definition of internal macros end
%%%%%%%%%%%%%%%%%%%%%%%%%%%%%%%%%%%%%%%%%%%%%%%%%%


%%%%%%%%%%%%%%%%%%%%%%%%%%%%%%%%%%%%%%%%%%%%%%%%%%
% Definition of external macros begin
%%%%%%%%%%%%%%%%%%%%%%%%%%%%%%%%%%%%%%%%%%%%%%%%%%

\def\setcourse#1{%
    \def\course{#1}%
    \def\headerPresent{}%
}

\def\setvariant#1{%
    \def\variant{#1}%
    \def\headerPresent{}%
}

\def\includenameinheader{%
    \def\printName{}%
    \def\headerPresent{}%
}

\def\includescoring{%
    \def\includeScoring{}%
    \def\headerPresent{}%
}

\def\printheader{%
    \setbox0=\hbox to \hsize{%
        \ifx\variant\undefined\else
            \rlap{Variant: {\bf\variant}}%
        \fi
        \ifx\course\undefined\else
            \hbox to \hsize{%
                \courseTitle
                \hss\course\hss
            }%
        \fi
        \ifx\printName\undefined\else
            \llap{%
                \hbox to 5cm{%
                    Name: \leaders\hbox{.}\hfil
                }%
            }%
        \fi
    }%
    % If scoring and there is no box above, don't vskip
    \ifdim\ht0=0pt\else
        \box0
        \ifx\includeScoring\undefined\else
            \vskip 1cm
        \fi
    \fi
    \ifx\includeScoring\undefined\else
        \printScoring
    \fi    
    \ifx\headerPresent\undefined\else
        \vskip 0.7cm
        \hrule
        \vskip 1cm
    \fi
}

\def\singlechoicequestion#1#2{%
    \def\questionType{single}%
    \question{#1}{#2}%
    \saveQuestionCnt
    \advance\questionCnt by 1
}

\def\multichoicequestion#1#2{%
    \def\questionType{multi}%
    \question{#1}{#2}%
    \saveQuestionCnt
    \advance\questionCnt by 1
}

\def\truefalsequestion#1#2{%
    \def\questionType{tf}%
    \question{#1}{#2}%
    \saveQuestionCnt
    \advance\questionCnt by 1
    \vbox{\bf\hskip 4pt T\hskip 6pt F}%
    \vskip 3pt
}

\def\choice#1{%
    \setbox0=\hbox{%
        \noindent\hskip\dimexpr\parindent+3pt
        \ifx\questionType\questionSingle
            \hbox to 12pt{\hss\singleChoiceBox\hss}%
            \hskip 3pt
        \fi
        \ifx\questionType\questionMultiple
            \hbox to 10pt{\hss\multiChoiceBox\hss}%
            \hskip 6pt
        \fi
        \ifx\questionType\questionTF
            \TFBox
            \hskip 6pt
        \fi
    }
    \hbox{%
        \copy0
        \vtop{%
            \noindent\hsize=\dimexpr\hsize-\wd0
            #1%
        }%
    }%
    \vskip 3pt
}

\def\endchoicequestion{%
    \vskip 12pt plus 1fil
    \def\questionType{none}%
}

\def\answerquestion#1#2#3{%
    \answerQuestion{#1}{#2}{#3}%
    \ifx\questionType\questionSub
        \advance\subquestionCnt by 1
    \else
        \saveQuestionCnt
        \advance\questionCnt by 1
    \fi
    \vskip 3pt
}

\def\subquestions#1#2{%
    \question{#1}{#2}%
    \def\questionType{sub}%
    \saveQuestionCnt
    \advance\questionCnt by 1
}

\def\endsubquestions{%
    \vskip 12pt plus 1fil
    \def\questionType{none}%
    \subquestionCnt=`a
}

%%%%%%%%%%%%%%%%%%%%%%%%%%%%%%%%%%%%%%%%%%%%%%%%%%
% Definition of external macros end
%%%%%%%%%%%%%%%%%%%%%%%%%%%%%%%%%%%%%%%%%%%%%%%%%%

% Example usage

\setcourse{BI-TEX}
\setvariant{A}
\includenameinheader
\includescoring

\printheader

\singlechoicequestion{Which of the folowing statements is correct? Only one answer is to be marked as this is a single choice question.}{}
\choice{User has full control over which element in the header will be present.}
\choice{The header may contain either all of the elements or none of them.}
\choice{The header may not be modified.}
\endchoicequestion

\multichoicequestion{Which of the folowing statements is correct? Please select all possibilities (anything between 0--4) as this is a multichoice question.}{2}
\choice{All the option boxes are nicely aligned, so they would not distract students from thinking.}
\choice{If the number of questions changes, the scoring table will automagically adapt to it.}
\choice{The positioning of the option boxes was not handled.}
\choice{User has a choice to show or not to show the number of points for a question. Both cases are handled.}
\endchoicequestion

\truefalsequestion{Please mark which of the following statements are true and which are false.}{}
\choice{If an aswer to a question is too long to fit on a page, it will not break and \TeX{} will print an error message on the screen.}
\choice{Line breaking is handled correctly and the answers will break with no problem. Also the checkboxes stay fixed to the first line of an answer.}
\choice{If a question itself is too long to fit on a single line, it will break correctly.}
\choice{If a question with free space under it is too big to fit on a page, it is handled correctly, so the question is placed on the next page.}
\endchoicequestion

\answerquestion{How many prime numbers are there?}{2}{1cm}

\answerquestion{How many prime numbers are there? Prove it. More space is left for that purpose.}{4}{4cm}

\subquestions{Here are two equations to be solved. They are shown as two subquestion to this question.}{5}
\answerquestion{Solve for $3x + 9 = 12x - 18$}{2}{2cm}
\answerquestion{Solve for $2x^2 + 6 = 4x + 6$}{3}{4cm}
\endsubquestions

\bye
